\chapter*{Appendices}
\addcontentsline{toc}{chapter}{Appendices}
\markboth{Appendices}{Appendices}

\section{Additional Tables and Figures}\label{sec:tables}

\begin{table}[ht]
    \centering
    \begin{tabular}{lllllll}
        \toprule
        $\sigma_y$ & $d_x$ & $d_y$ & SMCDiffOpt & DPS & $\Pi$IGD & TMPD \\
        \midrule
        \multirow[t]{6}{*}{0.0} & \multirow[t]{3}{*}{8} & 1 & 0.98 ± 0.44 & 8.22 ± 7.32 & 3.15 ± 2.58 & 3.5 ± 2.67 \\
         &  & 2 & 0.55 ± 0.43 & 0.41 ± 0.29 & 0.33 ± 0.27 & 0.44 ± 0.34 \\
         &  & 4 & 0.21 ± 0.07 & 0.12 ± 0.06 & 0.09 ± 0.03 & 0.08 ± 0.04 \\
        \cline{2-7}
         & \multirow[t]{3}{*}{80} & 1 & 0.75 ± 0.31 & 2.45 ± 1.79 & 3.18 ± 2.6 & 2.66 ± 1.47 \\
         &  & 2 & 1.22 ± 1.04 & 1.35 ± 1.21 & 0.33 ± 0.27 & 0.81 ± 0.68 \\
         &  & 4 & 0.26 ± 0.05 & 0.97 ± 0.86 & 0.08 ± 0.02 & 0.68 ± 0.44 \\
        \cline{1-7} \cline{2-7}
        \multirow[t]{6}{*}{0.1} & \multirow[t]{3}{*}{8} & 1 & 1.13 ± 0.51 & 8.18 ± 7.5 & 3.22 ± 2.67 & 3.11 ± 2.31 \\
         &  & 2 & 0.2 ± 0.07 & 0.38 ± 0.28 & 0.19 ± 0.13 & 0.43 ± 0.34 \\
         &  & 4 & 0.15 ± 0.05 & 0.16 ± 0.06 & 0.07 ± 0.01 & 0.06 ± 0.01 \\
        \cline{2-7}
         & \multirow[t]{3}{*}{80} & 1 & 1.25 ± 1.02 & 2.51 ± 2.11 & 2.93 ± 2.56 & 2.56 ± 1.18 \\
         &  & 2 & 0.45 ± 0.32 & 1.27 ± 1.14 & 0.62 ± 0.56 & 0.66 ± 0.54 \\
         &  & 4 & 0.23 ± 0.06 & 1.03 ± 0.9 & 0.08 ± 0.02 & 0.63 ± 0.34 \\
        \cline{1-7} \cline{2-7}
        \multirow[t]{6}{*}{1.0} & \multirow[t]{3}{*}{8} & 1 & 1.35 ± 1.0 & 6.62 ± 4.96 & 1.16 ± 0.72 & 1.35 ± 0.9 \\
         &  & 2 & 0.44 ± 0.28 & 2.92 ± 2.42 & 0.94 ± 0.89 & 1.44 ± 1.15 \\
         &  & 4 & 0.15 ± 0.03 & 1.19 ± 0.66 & 0.1 ± 0.03 & 0.45 ± 0.26 \\
        \cline{2-7}
         & \multirow[t]{3}{*}{80} & 1 & 1.69 ± 1.37 & 5.26 ± 4.4 & 1.2 ± 0.79 & 1.23 ± 0.53 \\
         &  & 2 & 0.89 ± 0.78 & 3.21 ± 2.95 & 1.68 ± 1.62 & 1.41 ± 1.16 \\
         &  & 4 & 1.12 ± 0.92 & 1.54 ± 0.99 & 0.89 ± 0.74 & 1.37 ± 0.67 \\
        \cline{1-7} \cline{2-7}
        \bottomrule
    \end{tabular}
    \caption{Sliced-Wasserstein distances for different $\sigma_y$.}
    \label{tab:gmm-sigma-split}
\end{table}

\newpage

\section{Proofs}\label{sec:proofs}

\begin{proof}[Proof of \autoref{prop:obs-gen}] \label{prf:obs-generation}
    Given:
    \begin{align*}
        \mathbf{X}_t \mid \mathbf{X}_0 = \mathbf{x}_0 &\sim \mathcal{N}(c_t\cdot\mathbf{x}_0, d_t^2\cdot\mathbf{I}_{d_x}) \\
        \mathbf{Y}_t \mid \mathbf{Y}_0 = \mathbf{y}_0 &\sim \delta(\mathbf{y}_t - c_t\cdot \mathbf{y}_0) \\
        \mathbf{Y}_0 \mid \mathbf{X}_0 = \mathbf{x}_0 &\sim \mathcal{N}(A\mathbf{x}_t, \sigma_y^2\mathbf{I}_{d_y})
    \end{align*}
    We have:
    \begin{align*}
        \EE{\mathbf{Y}_t \mid \mathbf{X}_0} &= \EE{\EE{\mathbf{Y}_t \mid \mathbf{Y}_0} \mid \mathbf{X}_0} \\
        &= \EE{c_t\cdot \mathbf{Y}_0 \mid \mathbf{X}_0} \\
        &= c_t\cdot A\mathbf{X}_0
    \end{align*}
    and
    \begin{align*}
        \VV{\mathbf{Y}_t \mid \mathbf{X}_0} &= \EE{\VV{\mathbf{Y}_t \mid \mathbf{Y}_0} \mid \mathbf{X}_0} + \VV{\EE{\mathbf{Y}_t \mid \mathbf{Y}_0} \mid \mathbf{X}_0} \\
        &= \VV{c_t\cdot \mathbf{Y}_0 \mid \mathbf{X}_0} \\
        &= c_t^2\sigma_y^2\cdot \mathbf{I}_{d_y}
    \end{align*}
    Hence:
    \begin{equation*}
        \mathbf{Y}_t \mid \mathbf{X}_0 = \mathbf{x}_0 \sim \mathcal{N}(c_t\cdot A\mathbf{x}_0, \sigma_y^2c_t^2\cdot \mathbf{I}_{d_y})
    \end{equation*}
    Further:
    \begin{align*}
        A\mathbf{X}_t \mid \mathbf{X}_0 = \mathbf{x}_0 &\sim \mathcal{N}(c_t\cdot A\mathbf{x}_0, d_t^2\cdot AA^\top) \\
        \implies \mathbf{Y}_t - A\mathbf{X}_t \mid \mathbf{X}_0 = \mathbf{x}_0 &\sim \mathcal{N}(0, \sigma_y^2c_t^2\cdot \mathbf{I}_{d_y} + d_t^2\cdot AA^\top) \\
        \implies \mathbf{Y}_t \mid \mathbf{X}_t = \mathbf{x}_t &\sim \mathcal{N}(A\mathbf{x}_t, \sigma_y^2c_t^2\cdot \mathbf{I}_{d_y} + d_t^2\cdot AA^\top)
    \end{align*}

\end{proof}

\begin{proposition}[Gaussian-Gaussian Exactness for Linear Observations] \label{prop:gaussian-exact}
    Suppose
    \begin{align*}
        \mathbf{X}_{t} \mid \mathbf{X}_{t+1} &\sim \mathcal{N}\left( \mathbf{\mu}_{t+1}(\mathbf{x}_{t+1}), \Sigma_{t+1} \right)  \\
        \mathbf{Y}_{t} \mid \mathbf{X}_{t} = \mathbf{x}_{t} &\sim \mathcal{N}\left( A\mathbf{x}_{t}, \Xi_{t} \right)
    \end{align*}
    Consider some proposal:
    \begin{align*}
        p_{t}(\mathbf{x}_{t} \mid \mathbf{y}_{t}, \mathbf{x}_{t+1}) \propto p_{t}(\mathbf{x}_{t} \mid \mathbf{x}_{t+1})g_{t}(\mathbf{y}_{t} \mid \mathbf{x}_{t})
    \end{align*}
    Then:
    \begin{align*}
        \mathbf{Y}_{t} \mid \mathbf{X}_{t+1} = \mathbf{x}_{t+1} \sim \mathcal{N}\left( A\mu_{t+1}, \Xi_{t} + A\Sigma_{t+1}A^\top \right) 
    \end{align*}
\end{proposition}
\begin{proof}
    By Gaussian conjugacy, it immediately follows that:
    \begin{align*}
        \mathbf{X}_{t} \mid \mathbf{Y}_{t} = \mathbf{y}_{t}, \mathbf{X}_{t+1} = \mathbf{x}_{t+1} &\sim \mathcal{N}\left( \mu_{t+1}^P, \Sigma_{t+1}^P \right) 
    \end{align*}
    where
    \begin{align*}
        \mu_{t+1}^P &= \Sigma_{t+1}^P\left( \Sigma_{t+1}^{-1}\mu_{t+1}(\mathbf{x}_{t+1}) + A^\top\Xi_{t}^{-1}\mathbf{y}_{t} \right) \\
        \Sigma_{t+1}^P &= \left( \Sigma_{t+1}^{-1} + A^\top \Xi_{t}^{-1}A \right)^{-1}
    \end{align*}
    The normalizing constant of the proposal is:
    \begin{align*}
        \int p_{t}(\mathbf{x}_{t} \mid \mathbf{x}_{t+1})g_{t}(\mathbf{y}_{t} \mid \mathbf{x}_{t}) \, d\mathbf{x}_{t} &= g_{t}(\mathbf{y}_{t} \mid \mathbf{x}_{t+1})
    \end{align*}
    Note that:
    \begin{align*}
        \mathbb{E}\left\{ \mathbf{Y}_{t} \mid \mathbf{X}_{t+1} \right\} &= \mathbb{E}\left\{\mathbb{E}\left\{\mathbf{Y}_{t} \mid \mathbf{X}_{t}\right\} \mid \mathbf{X}_{t+1}\right\} &\dots\text{tower property} \\
        &= \mathbb{E}\left\{A\mathbf{X}_{t} \mid \mathbf{X}_{t+1}\right\} \\
        &= A\mathbb{E}\left\{\mathbf{X}_{t} \mid \mathbf{X}_{t+1}\right\} \\
        &= A\mu_{t+1}(\mathbf{X}_{t+1}) \\
        \mathbb{V}\left\{\mathbf{Y}_{t} \mid \mathbf{X}_{t+1}\right\} &= \mathbb{E}\left\{\mathbb{V}\left\{\mathbf{Y}_{t} \mid \mathbf{X}_{t}\right\} \mid \mathbf{X}_{t+1}\right\} + \mathbb{V}\left\{\mathbb{E}\left\{\mathbf{Y}_{t} \mid \mathbf{X}_{t}\right\} \mid \mathbf{X}_{t+1}\right\} &\dots \text{total variance} \\
        &= \mathbb{E}\left\{\Xi_{t} \mid \mathbf{X}_{t+1}\right\} + \mathbb{V}\left\{A\mathbf{X}_{t} \mid \mathbf{X}_{t+1}\right\} \\
        &= \Xi_{t} + A\mathbb{V}\left\{\mathbf{X}_{t} \mid \mathbf{X}_{t+1}\right\}A^{\top} \\
        &= \Xi_{t} + A\Sigma_{t+1}A^{\top}
    \end{align*}
    Since $\mathbf{Y}_{t}$ is an affine transformation, it follows then that:
    \begin{align*}
        \mathbf{Y}_{t} \mid \mathbf{X}_{t+1} = \mathbf{x}_{t+1} \sim \mathcal{N}\left( A\mu_{t+1}, \Xi_{t} + A\Sigma_{t+1}A^\top \right) 
    \end{align*}
\end{proof}

\newpage

\section{Further Experimental Details} \label{sec:experimental-extra}

\paragraph{Gaussian Mixture Model} We initially aimed to include MCGdiff
\parencite{cardosoMonteCarloGuided2023} in our comparison. However, we were unable to reproduce
their results, even using the code on their repository. Indeed, we found that the resulting sliced
Wasserstein distances from MCGdiff were almost uniformly larger (worse) than those of
\text{SMCDiffOpt} on each experimental run (i.e. given identical setups and pseudo-RNG seeds). This
is doubly unexpected since the results we were seeing don't align with those found in their paper,
and since theoretically their proposal should be more optimal than ours for this particular task;
if we take the values they give in Table 1 of their paper and compare them with those in
\autoref{tab:gmm} we see this is indeed the case. Given the time constraints, the exhaustion of all
obvious potential author-errors, and assuming the validity of their results, we opted to omit our
results from a comparison. For posterity, we provide in \autoref{tab:gmm-mcg} the results including
MCGdiff but note with caution that we lack confidence in such figures.

\begin{table}[ht]
    \centering
    \begin{tabular}{llllllll}
        \toprule
        $\sigma_y$ & $d_x$ & $d_y$ & SMCDiffOpt & MCGDiff & DPS & $\Pi$IGD & TMPD \\
        \midrule
        \multirow[t]{6}{*}{0.0} & \multirow[t]{3}{*}{8} & 1 & 0.98 ± 0.44 & 0.95 ± 0.44 & 8.22 ± 7.32 & 3.15 ± 2.58 & 3.5 ± 2.67 \\
         &  & 2 & 0.55 ± 0.43 & 0.39 ± 0.33 & 0.41 ± 0.29 & 0.33 ± 0.27 & 0.44 ± 0.34 \\
         &  & 4 & 0.21 ± 0.07 & 0.15 ± 0.09 & 0.12 ± 0.06 & 0.09 ± 0.03 & 0.08 ± 0.04 \\
        \cline{2-8}
         & \multirow[t]{3}{*}{80} & 1 & 0.75 ± 0.31 & 1.06 ± 0.77 & 2.45 ± 1.79 & 3.18 ± 2.6 & 2.66 ± 1.47 \\
         &  & 2 & 1.22 ± 1.04 & 2.31 ± 2.21 & 1.35 ± 1.21 & 0.33 ± 0.27 & 0.81 ± 0.68 \\
         &  & 4 & 0.26 ± 0.05 & 2.22 ± 1.92 & 0.97 ± 0.86 & 0.08 ± 0.02 & 0.68 ± 0.44 \\
        \cline{1-8} \cline{2-8}
        \multirow[t]{6}{*}{0.1} & \multirow[t]{3}{*}{8} & 1 & 1.13 ± 0.51 & 0.79 ± 0.62 & 8.18 ± 7.5 & 3.22 ± 2.67 & 3.11 ± 2.31 \\
         &  & 2 & 0.2 ± 0.07 & 1.15 ± 1.09 & 0.38 ± 0.28 & 0.19 ± 0.13 & 0.43 ± 0.34 \\
         &  & 4 & 0.15 ± 0.05 & 0.25 ± 0.12 & 0.16 ± 0.06 & 0.07 ± 0.01 & 0.06 ± 0.01 \\
        \cline{2-8}
         & \multirow[t]{3}{*}{80} & 1 & 1.25 ± 1.02 & 1.26 ± 0.87 & 2.51 ± 2.11 & 2.93 ± 2.56 & 2.56 ± 1.18 \\
         &  & 2 & 0.45 ± 0.32 & 3.06 ± 2.93 & 1.27 ± 1.14 & 0.62 ± 0.56 & 0.66 ± 0.54 \\
         &  & 4 & 0.23 ± 0.06 & 3.34 ± 3.01 & 1.03 ± 0.9 & 0.08 ± 0.02 & 0.63 ± 0.34 \\
        \cline{1-8} \cline{2-8}
        \multirow[t]{6}{*}{1.0} & \multirow[t]{3}{*}{8} & 1 & 1.35 ± 1.0 & 1.14 ± 0.7 & 6.62 ± 4.96 & 1.16 ± 0.72 & 1.35 ± 0.9 \\
         &  & 2 & 0.44 ± 0.28 & 1.11 ± 1.04 & 2.92 ± 2.42 & 0.94 ± 0.89 & 1.44 ± 1.15 \\
         &  & 4 & 0.15 ± 0.03 & 0.12 ± 0.06 & 1.19 ± 0.66 & 0.1 ± 0.03 & 0.45 ± 0.26 \\
        \cline{2-8}
         & \multirow[t]{3}{*}{80} & 1 & 1.69 ± 1.37 & 1.49 ± 0.75 & 5.26 ± 4.4 & 1.2 ± 0.79 & 1.23 ± 0.53 \\
         &  & 2 & 0.89 ± 0.78 & 2.08 ± 1.97 & 3.21 ± 2.95 & 1.68 ± 1.62 & 1.41 ± 1.16 \\
         &  & 4 & 1.12 ± 0.92 & 4.52 ± 2.55 & 1.54 ± 0.99 & 0.89 ± 0.74 & 1.37 ± 0.67 \\
        \cline{1-8} \cline{2-8}
        \bottomrule
    \end{tabular}
    \caption{Sliced-Wasserstein distances for GMM experiment with MCGdiff included. These results
    do not align with those in \textcite{cardosoMonteCarloGuided2023}, despite running exactly
    the script in their repository.}
    \label{tab:gmm-mcg}
\end{table}

\paragraph{Branin Optimization} \textbf{TODO}: Model architecture description...

\newpage

\section{Additional Background and Related Works} \label{sec:extra}

\paragraph{Diffusion Models}

\begin{remark}[SDE Representation] \label{rem:sde-rep}
    The DDPM approach corresponds to a discretized time rescaled Ornstein-Uhlenbeck process
    \parencite{boysTweedieMomentProjected2023,songScoreBasedGenerativeModeling2021}:
    $$
    d\mathbf{x}_t = -\frac{1}{2}\beta(t)\mathbf{x}_t dt + \sqrt{\beta(t)}d\mathbf{w}_t
    $$
    and is often referred to as the variance-preserving SDE
    \parencite{songScoreBasedGenerativeModeling2021}. Going forwards, we will stick to the Markov
    (discretised) representation, but ultimately they are equivalent when it comes to sampling.
    This point is worth noting though as the SDE representation is what analytically justifies the
    backwards process (see \autoref{ftnt:sde-rep}).
\end{remark}

\begin{remark}[DDIM Representation] \label{rem:ddim}
    Under DDIM:
    \begin{equation*}
        u_t = \sqrt{\alpha_{t-1}} \quad
        v_t = -\sqrt{1 - \overline{\alpha}_{t-1} - \sigma_t^2}\sqrt{1 - \overline{\alpha}_t} \quad
        w_t = \sigma_t
    \end{equation*}
    with $\{\sigma_t\}_{t=1}^T$ an arbitrary conditional variance sequence. It's common to consider:
    \begin{equation*}
        \sigma_t(\eta) = \eta\sqrt{\frac{1 - \alpha_{t-1}}{1 - \alpha_t}}\sqrt{1 - \frac{\alpha_t}{\alpha_{t-1}}},\quad \eta \in [0,1]
    \end{equation*}
    with $\eta=1$ ultimately reducing $u_t, v_t, w_t$ to the DDPM values, and $\eta=0$ corresponding
    to deterministic generation \parencite{songDenoisingDiffusionImplicit2020}.
\end{remark}

\begin{remark}[Time Respacing]
    One of the remarkable features of the DDIM algorithm is it enabling \emph{time re-spacing}
    whereby we can sample from the backwards process in fewer time-steps. In this paper, we don't
    consider such re-spacing though in principle our methodology does not prohibit it.
\end{remark}

\paragraph{FPS-SMC}

Rather than \autoref{eq:obs-gen}, \textcite{douDiffusionPosteriorSampling2023} instead considers
a ``shared-noise'' or ``duplex'' diffusion:

\begin{proposition}[Obvservation Diffusion] \label{prop:obs-diffusion}
    \textcite[Proposition B.1]{douDiffusionPosteriorSampling2023}.
    Let $\mathbf{x}_t$ be generated according to the forward marginal of some diffusion process.
    Let $\mathbf{y} = \mathbf{y}_0$ be some (linear) measurement we have about the clean sample,
    $\mathbf{x}_0$. Construct a sequence $\{\mathbf{y}_t\}_{t=1}^T$ by:
    \begin{equation*}
        \mathbf{y}_t = a_t\cdot \mathbf{y}_{t-1} + b_t\cdot A\epsilon_t,\quad \epsilon_t \sim \mathcal{N}(0, \mathbf{I}_{d_y})
    \end{equation*}
    where $\epsilon_t$ is the \emph{same} as used for forward noising $\mathbf{x}_0$ at time-step $t$.
    It follows that:
    \begin{equation}
        \mathbf{Y}_t \mid \mathbf{X}_t = \mathbf{x}_t \sim \mathcal{N}(A\mathbf{x}_t, \sigma_y^2c_t^2\cdot \mathbf{I}_{d_y}) \label{eq:obs-likelihood-dou}
    \end{equation}
    with $g(\mathbf{y}_0 \mid \mathbf{x}_0) = g(\mathbf{y} \mid \mathbf{x}_0) = \mathcal{N}(\mathbf{y}; A\mathbf{x}_0, \sigma_y^2\cdot \mathbf{I}_{d_y})$
    exactly the measurement density.
\end{proposition}

Note because of the noise-sharing, we get cancellation in the covariance of the likelihood, dropping
the $d_t^2\cdot AA^\top$ term as in \autoref{eq:obs-likelihood}.

Based on \autoref{prop:obs-diffusion} and \textcite[Remark B.1]{douDiffusionPosteriorSampling2023},
we can generate some $\{\mathbf{y}_t\}_{t=0}^T$ either forwards or \emph{backwards}.
The latter follows because where in the backwards process for $\mathbf{x}_t$ we use an estimate of
$\mathbf{x}_0$ (via Tweedie's formula), for the observations we have $\mathbf{y}_0$, and an
expression can be analytically derived.
