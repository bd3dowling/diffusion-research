% Generates plagiarism declaration
\declarationname{Brendan Dowling}
\declarationdate{\today}
\declaration 

\begin{acknowledgements}
    I want to give special thanks to my supervisor, Dr Deniz Akyildiz, who has provided excellent
    guidance over the course of this work. The dedication of his time towards answering my questions
    and suggesting avenues to explore went beyond expectations and was greatly appreciated.
    I also want to thank Benjamin Boys for meeting with me several times, providing insightful
    and guiding discussions related to this research, and for providing access to his code.
    I also want to give special thanks to my family for their continued support of my studies and
    for providing me the resources needed to excel.
    Lastly, I want to thank my fianc\'ee for her support over this challenging past year; without
    her this thesis could not have happened.
\end{acknowledgements}

\begin{abstract}
    Diffusion models have have been shown to learn underlying data distributions, even in
    high-dimensional settings as demonstrated by their state-of-the-art performance on image, video
    and molecular synthesis tasks. From a Bayesian perspective, this makes them suitable as prior
    models. Recent research has explored how to accurately \emph{guide} the diffusion process for
    posterior sampling. Such methods have primarily focused on solving ill-posed inverse problems by
    approximating the time conditional score function to use in the reverse sampling steps. In this
    paper, we present a novel and general framework, \texttt{SMCDiffOpt}, for guiding diffusion models
    with sequential Monte Carlo methods. The framework considers inverse problems as a special case
    of optimisation, where we've reframed the optimisation task as a sampling problem from an
    annealed Boltzmann-Gibbs distribution defined by the objective, using the diffusion model as
    a prior. This framework can be applied even in settings where gradients are not available,
    making it both computationally efficient and applicable in black-box optimisation settings.
    Relatedly, this framework does not necessitate conditional score approximations though can
    flexibly accommodate them for better sampling efficiency. We demonstrate the efficacy of the
    algorithm through experiments on synthetic and real-world inverse problems and optimisation
    tasks.
\end{abstract}

\tableofcontents
\markboth{}{}

% Switch from Roman to Arabic numbering for main part of report. Do not modify.
\mainmatter
