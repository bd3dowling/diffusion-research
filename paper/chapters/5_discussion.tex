\chapter{Discussion} \label{chap:discussion}

% \paragraph{Tips for the Discussion section.} 

% \begin{itemize}
%     \item Begin your Discussion with a summary and the main findings of your research in 2-5 sentences.
%     \item Describe the implications of your main findings in the context of existing work concisely and precisely using scientific language. 
%     \item Describe the limitations in your statistical methods and your main findings. Be honest about the limitations in your approach, and substantiate what could have been done differently as needed. Explain if your main findings are robust or sensitive to these limitations.
%     \item Avoid Subsections and Subsubsections in the Discussion.
%     \item In the last paragraph, conclude your report with a pitch using plain language that summarises the key implications of your research in the context of previous work. Write this last paragraph for the Imperial Press Office or journalists as audience.
%     \item Aim for approximately 1-3 pages, similar in style to a general science or statistics research paper.
% \end{itemize}

\textbf{TODO; appx 1.5 pages}

\begin{itemize}
    \item Re-describe benefits of approach and relative contribution of work.
    \item Re-describe limitations/shortcomings of approach.
    \item Discuss future work; ablation study, incorporation of gradients, non-linear inverse problems, apply to imagery or other high-dim inverse problems, full black-box benchmark
    \item Summary concluding remarks.
\end{itemize}

\section*{Endmatter}

All code for the research is openly available on a
\href{https://github.com/bd3dowling/diffusion-research}{GitHub repository}.
All results in this paper can be easily reproduced by following the instructions of said
repository's \texttt{README}. The SuperConductor experiment code is separately available on
\href{https://github.com/bd3dowling/superconductor}{this repository} due to complex dependency
requirements. Code was run on an NVIDIA RTX3080.

Black-box data and oracle models are available originally through the
\href{https://github.com/brandontrabucco/design-bench}{\texttt{design-bench} repository}.
