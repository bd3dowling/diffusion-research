\chapter{Discussion} \label{chap:discussion}

Summary:
- Intro SMCDiffOpt, an SMC-based diffusion-based optimiser

Benefits:
- Performance
- General optimisation
- Flexible
- Extensible

Drawbacks:
- Slow sampling time due to diffusion.
- Not necessarily valid configs.
- Needs multiple particles.
- Requires approximations for non-linear inverse tasks.
- Note that this is true in DPS too technically.
- Requires tuning, particularly of $\gamma(t)$ and resampling rate.

Future work:
- More complete comparison of performance versus other SMC inverse solvers.
- Apply to image data and contrast resulting FIDs with other inverse solvers.
- Ablation study on performance as a function of:
- Number of particles
- Noising schedules
- Trial of different proposals for inverse problems.
- Examine incorporation of gradients via nudging.
- Examine performance on non-linear inverse problems based on approximations.
- Examine performance with time-re-spacing DDIM.
- Complete benchmark on all tasks in design-bench benchmark.
- Score for diversity...
- Theoretical guarantees.

Concluding remarks
- In the last paragraph, conclude your report with a pitch using plain language that summarises the key implications of your research in the context of previous work. Write this last paragraph for the Imperial Press Office or journalists as audience.


\section*{Endmatter}

All code for the research is openly available on a
\href{https://github.com/bd3dowling/diffusion-research}{GitHub repository}.
All results in this paper can be easily reproduced by following the instructions of said
repository's \texttt{README}. The SuperConductor experiment code is separately available on
\href{https://github.com/bd3dowling/superconductor}{this repository} due to complex dependency
requirements. Code was run on an NVIDIA RTX3080.

Black-box data and oracle models are available originally through the
\href{https://github.com/brandontrabucco/design-bench}{\texttt{design-bench} repository}.
