\chapter{Discussion} \label{chap:discussion}

We proposed \texttt{SMCDiffOpt}, a novel approach for general optimisation based on sequential
Monte Carlo guided diffusion models. The method operates in zero-shot, taking any pre-trained
unconditional diffusion model and enabling its usage for optimisation and inverse problem tasks.
Existing similar diffusion-based methods have been sub-divided into those which solve inverse problems
\parencite{chungDiffusionPosteriorSampling2022,song2023pseudoinverseguided,boysTweedieMomentProjected2023,cardosoMonteCarloGuided2023,douDiffusionPosteriorSampling2023,wuPracticalAsymptoticallyExact2023}
and those which solve optimisation tasks
\parencite{kongDiffusionModelsConstrained2024,krishnamoorthyDiffusionModelsBlackBox2023,guoGradientGuidanceDiffusion2024}.
Our approach bridges these domains by framing optimisation as a posterior sampling task and inverse
problem solving as a special case of optimisation. For inverse problems, we described how the
method can be applied by constructing a sequence of aligned measurements and likelihoods, taking
maximisation of these likelihoods as the optimisation objective. We tested \texttt{SMCDiffOpt} on a
variety of tasks and showed its ability to yield state-of-the-art performance.

Our approach offers several advantages over existing methods. It yields asymptotically exact
posterior samples as the number of particle increases, giving significant performance gains over
independent samplers in suitable settings. It also operates without requiring access to gradients,
making it suitable for black-box optimisation tasks. It's also flexible, accommodating different
proposals and incorporation of gradients via nudging. It's also extensible, supporting intermediate
or terminal steps for improved sampling.

\texttt{SMCDiffOpt} is not without some limitations. The method inherits the slow sampling speed of
diffusion models compared to other generative approaches. More importantly, the method requires
multiple particles to work effectively --- in cases of certain inverse tasks more efficient
proposals may mitigate this drawback but never fully remove it. Additionally, in the context of
general optimisation, the final particles may not strictly adhere to valid configurations --- this
problem is illustrated in \textcite{guoGradientGuidanceDiffusion2024}. In many settings, such as
image generation, particles being `close' to the data manifold may be sufficient; in others, such
as chemical generation, the requirements may be more stringent, with the results of
\texttt{SMCDiffOpt} requiring further analysis and modification. \texttt{SMCDiffOpt} also offers
only approximate methods for solving non-linear inverse tasks (see \autoref{sec:non-linear}), though
it should be noted this is true even of DPS \parencite{chungDiffusionPosteriorSampling2022,boysTweedieMomentProjected2023}.
Finally, in certain tasks, tuning of $\gamma(t)$ and the resampling rate (i.e. minimum ESS) may be
challenging to tune.

There are several potential avenues for future work. First, a comparison of the relative performance
of \texttt{SMCDiffOpt} versus other SMC-based samplers
\parencite{cardosoMonteCarloGuided2023,douDiffusionPosteriorSampling2023,wuPracticalAsymptoticallyExact2023}
for linear inverse problems would be insightful (see \autoref{sec:experimental-extra}), including
plugging their twisted proposals in to \autoref{alg:smc-opt} and measuring the relative efficiency
gains. Relatedly, our work lacks an ablation study on the impact of the number of particles, the
noising schedule, the resampling rate and the annealing schedule which are the key variables in
\autoref{alg:smc-opt}. Examining the performance and viability of our method on higher-dimensional
settings, such as image generation, would also be fruitful. Similarly for non-linear inverse
problems using the approximate methods described in \autoref{sec:non-linear}. An examination
of \texttt{SMCDiffOpt}'s performance under a \hyperref[rem:time-respacing]{time-respaced} diffusion
model would be insightful and if positive suggest better applicability under query budget
constraints for black-box optimisation. Fully benchmarking \texttt{SMCDiffOpt} on the
\texttt{design-bench} black-box optimisation tasks must be carried out to validate the
state-of-the-art performance of the algorithm over previous works. Devising a new scoring method
which accounts for diversity of optimised samples should be considered too, since this is one of the
key benefits of \texttt{SMCDiffOpt}. Lastly, an analysis of the theoretical guarantees of running
\autoref{alg:smc-opt} should be conducted.

\section*{Endmatter}

All code for the research is openly available on a
\href{https://github.com/bd3dowling/diffusion-research}{GitHub repository}.
All results in this paper can be easily reproduced by following the instructions of said
repository's \texttt{README}. The SuperConductor experiment code is separately available on
\href{https://github.com/bd3dowling/superconductor}{this repository} due to complex dependency
requirements. Code was run on an NVIDIA RTX3080.

Black-box data and oracle models are available originally through the
\href{https://github.com/brandontrabucco/design-bench}{\texttt{design-bench} repository}.
